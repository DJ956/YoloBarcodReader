%評価・考察
本章では、V字モデルに従って実装したシステムを評価し、本システムに対する利点と問題点について述べる。
このシステムの利点は、Yoloを使用することで高い精度でバーコード領域特定を行えることである。複数の商品を、同時に読み取る場合であっても、Yoloを使用すれば容易にバーコードの検知を行うことができる。バーコード識別可能なモデルの学習も画像2000枚程度で、高い精度を出すことができる。

\begin{table}[htb]
\begin{center}
\caption{実行環境}
\begin{tabular}{|c|c|} \hline
Webカメラと対象の距離 & 精度\\ \hline
10cm & 98~99\% \\ \hline
15cm & 92~93\% \\ \hline
20cm & 95~96\% \\ \hline
\end{tabular}
\label{accurancy}
	\end{center}
\end{table}

上の表\ref{accurancy}より、対象から距離が離れても極めて高い精度を保っていることがわかる。Yoloの学習にかかった時間もおよそ10時間ほどであり、比較的短時間で学習が終了した。高い精度を出すことができる理由として、学習する際に各バーコード画像がもつ特徴の差が少ないということと、判別する対象が1つだけであることが挙げられる。この特徴の差が少ないというのは、機械学習において精度向上と学習時間短縮に繋がる\cite{deep_learning}。例えば、猫の識別の場合、猫にも様々な種類があり、精度を高めるには様々な種類の猫のサンプルを集める必要がある。しかし、フォーマットの定められているバーコードなら、多くのサンプルを集める必要がない。また、判別する対象が1つだけであることは、学習時間短縮につながる。判別する種類が多くなると一般的\cite{deep_learning}に、機械学習で物体を判別する際により複雑なネットワークと学習用の画像データを多く必要とする。また、高性能なGPUが必要になる。

一般的な大規模店舗では、数万種類の商品を扱っていることもある。もし、Yoloをバーコード識別に使用するのではなく、商品個別の判別に用いる場合、数万種類を高い精度で判別できるネットワークを構築する必要がある。さらに、それの学習を行うことのできる性能の機材も必要となる。もし、そのようなネットワークを実現できたとしても、新しい商品が追加されるとその商品の判別を行えるように画像データを収集し、再学習する必要がある。今の技術及びコストを考慮すると、これらを行うのは非現実的である。そこで、どの商品でもほとんどフォーマットが変わらないバーコードの識別を行うことにした。

今回実装と検証を行って、一番の問題点となったのがWebカメラが撮影した商品の距離によって検知率が大きく異なる点である。今回使用したWebカメラはロジクール製のC615モデルである。このモデルを選ぶ前に同メーカーのC270nモデルを使用していたが、商品から5cm程の近距離でバーコードを撮影しても、バーコード番号の識別が不可能であった。しかしながら、Yoloによる領域検知は行うことができた。C270nモデルの画質では番号判別ができなかったため、C615モデルに変更した。テストを行ったところ、約10cmの範囲の近距離ならば、pyzbarで正確に識別できることが、判明した。
さらにテスト行い距離を10cm以上伸ばしたところ、画像のバーコードがぼやけてバーの境界があやふやになっている状態だったため、検知しないことが分かった。
バーの境界を明確にするために画像に対して2値化処理とノイズ除去を行ったが、結果は変わらなかった。代わりに、iPhone5cを使用して15cmの距離で撮影した画像でテストしたところ、pyzbarでの識別を行うことができた。本システムはWebカメラの画質によって識別精度が大きく影響されると結論付けられる。