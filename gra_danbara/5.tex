%評価・考察
 本章ではV字モデルに従って実装したシステムを評価することにより、本システムに対する利点と問題点について述べる。

 このシステムの利点としてYoloを使用することで高い精度でバーコード領域特定ができることである。複数の商品を同時に読み取りたい場合であってもYoloを使用すれば容易にバーコードの検知を行うことができる。バーコード識別ができるモデルの学習も画像2000枚程度での学習で高い精度が出る。

\begin{table}[htb]
\begin{center}
\caption{実行環境}
\begin{tabular}{|c|c|} \hline
Webカメラと対象の距離 & 精度\\ \hline
10cm & 98~99\% \\ \hline
15cm & 92~93\% \\ \hline
20cm & 95~96\% \\ \hline
\end{tabular}
\label{accurancy}
	\end{center}
\end{table}

 上の表\ref{accurancy}を見てもわかる通り対象から距離が離れても極めて高い精度を保っていることがわかる。学習にかかった時間もおよそ10時間ほどである。これの高い精度を出すことができる理由として、判別する対象が1つだけである点と学習するバーコード画像自体の違いが少ないという点もある。この違いが少ないというのは例えば猫の識別の場合猫といっても様々な種類があり精度を高めるには様々な種類の猫のサンプルを集める必要があるが、フォーマットの定められているバーコードなら多くのサンプルを集める必要がない。もう1つの判別する対象が1つだけであることで得られる利点としては、機械学習で物体を判別する際に判別する種類が多くなると一般的により複雑なネットワークと学習用の画像データを必要としまた、強力なGPUが必要になる。一般的な大型販売店などでは数万種類の商品を扱っている。もしYoloをバーコード識別に使用するのではなく商品個別の判断をする場合、数万種類を高い精度で判別できるネットワークを構築する必要がありまたそれの学習を行える性能の機材も必要となる。またそのような判別機を実現できたとしても新しい商品が追加されるとその商品の判別を行えるように画像データを収集し、再学習する必要がある。今の技術ではこれらを行うのは非現実的である。そこでどの商品でもほとんどフォーマットが変わらないバーコードの識別を行うことにした。

 今回実装と検証を行って一番の問題点となったのがWebカメラが撮影した商品の距離によって検知率が大きく異なる点である。今回使用したWebカメラはロジクール製のC615を使用している。こちらのモデルを選ぶ前に同メーカーのC270nを使用していたが5cm程の近距離でバーコードを撮影してもバーコード番号の識別が不可能であった。ここでいう識別はpyzbarによる番号の識別でありYoloによる領域判定は検知していた。C270nモデルの画質では判別が不可能と判明したためC615モデルに変更したところ約10cmの範囲の近距離ならばpyzbarで正確にすべて識別することがテストを行い判明した。しかし、C615モデルで10cm以上離すと検知しなくなった。画像に対して2値化処理とノイズ除去を行ったが、元の画像のバーコードがぼやけてバーの境界があやふやになっている状態だったため結果は変わらないという結果に至った。代わりにiPhone5cで15cmの距離で撮影したバーコードを映した動画をテストしたところpyzbarでの識別をある程度行うことができたため、本システムはWebカメラの画質によって大きく識別精度が変わる問題点がある。