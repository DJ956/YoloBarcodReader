%第4章:実験結果・考察
本章ではV字モデル開発に従って実装を行った。ここでは具体的な実装方法について述べる。
\section{実装}
 実装に至っては2人のグループで行いエッジ処理側は真鍋樹が行った。ここではサーバ側の実装について述べる。ここではエッジ処理側のことをクライアントと呼ぶ。実装に使用したプログラム言語はクライアント、サーバどちらもPython3を使用した。

\subsection*{サーバ通信}
 クライアントとのデータのやり取りを含めた連携には通信処理が必要不可欠になる。クライアントはセンサが反応したらカメラを起動し複数の画像を撮影する。つまりサーバに送られる画像データは1回につき1枚ではない。つまりクライアントが画像データを送信するデータサイズが不明のため、サーバ側では送信されたデータのどこからどの部分が1つの商品画像データになるのか判断できない。そこで、クライアントは画像データを送信する前に画像データの合計サイズを送信する。サーバは合計サイズ分だけデータを受信すれば続けて2回目のデータ送信が来ても1回目のデータと区別することができる。
クライアントから送られてきた画像データはバイナリ形式になっているのでOpenCVのフォーマットに変換しなおしている。

\subsection*{Yoloによるバーコード領域特定}
 バーコード番号識別のために使用しているpyzbar\cite{pyzbar}.は画像データに占められるバーコードの割合が少ない場合識別精度が下がる。また、検証した結果画像に複数のバーコードが写っていた場合1つしか認識しない場合がある。そこでYoloを使用してバーコード領域を切りとることで複数のバーコードがあっても認識できるようにサーバのプログラムに組み込んだ。

\subsection*{DBを使用した商品情報の管理}

\section{検証}