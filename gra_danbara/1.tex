%まえがき
本論文では、RaspberryPiと各種センサとサーバを組み合わせた商品識別システムを開発した。現在の日本は少子高齢化の流れで人的資源が減少している。総務省の平成28年度の人口調査の調べでは2030年には1億1662万人に減少し、2050年には人口が1億人下回ると見込まれる\cite{population}。人的資源の減少という問題は社会全体に関わることである。人手不足問題解消のために、システムの無人化が急務となっている。

サービス業者にも人手不足の問題が深刻化おり、セルフレジの導入が進んでいる。だが、セルフレジは導入コストが高いという問題点がある。セルフレジは、登録機と精算機を合わせて平均で300万を超える\cite{self_register}。セルフレジの導入は、人手不足問題を抱える店舗においてメリットも大きいが負担も大きい。特に小規模な店舗の導入における負担が大きく、既存のセルフレジより安価で導入しやすいシステムの開発が求められている。

これらの問題を解決するために、資金力を持たない店舗でも導入しやすく、安価で人手のかからないシステムの構築を本研究の目的とした。我々はバーコードスキャン技術を利用した商品識別システムを提案する。

本研究では商品の識別から決算までのシステム構築を、V字モデルに従い、グループ(段原丞治、真鍋樹)で研究を行った。
要求定義や設計はUML図を用いて定義し、それをもとに開発を行う。実装に関してはエッジ処理側とサーバ処理側で担当を分けた。エッジ側のハードウェアはRaspberryPiを使用した。サーバは画像処理をYolo(画像から物体を識別する機械学習ネットワーク)\ref{yolo}やOpenCV(画像処理ライブラリ)\ref{opencv}等を利用して行う。
本論文は以下のような構成をとる。第2章では、本研究で使用した用語、技術の解説を述べる。第3章では、UML図を用いてシステムの要求定義、設計、検証項目を述べる。第4章では、実装内容と検証結果を示す。第5章では実装したシステムの評価及び考察を述べる。第6章では本研究のまとめを示す。