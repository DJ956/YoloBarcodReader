%第3章
%問題解決方法の選定:表を作成して問題解決方法を比較評価する.評価の理由と根拠をあげる(要データ,関連文献)
%解決方法の実行可能性の検証:自分がこの解決方法を選択した理由を述べる(3つの観点については本参照)


本章では,V字モデルによる商品識別システムの要求定義,基本設計,詳細設計について述べる.商品識別システムでは,全ての設計を通して,下記の3点を基本の評価軸とした.

\begin{itemize}
\item 従来のセルフレジよりコストは抑えられるか.
\item 既存の中小店でも導入が容易か.
\item 従来のセミセルフレジより簡単な動作で決済まで行えるか.
\end{itemize}


上述の「従来のレジよりコストは抑えられるか.」という評価軸には,カゴ90台を導入するコストと,登録機1台1,875,000円と精算機$2,750,000円\times7台$として,合わせておよそ21,125,000円\cite{super}とレジの店員分の人件費を合わせたコストを比べた際,よりコストを抑えられるかという意味を含んでいるとする.また,上述の「従来のレジより簡単な動作での決済まで行えるか.」という評価軸は,従来のレジの店員のように,商品を手に取り,バーコードリーダで商品のバーコードを読み取り,カゴへ入れるという動作と全く同じ動作をしないという意味を含んでいるとする.


本章の構成について,3.1節ではユースケース図を用いて,商品識別システムの要求定義を述べる.3.2節ではクラス図を用いて,商品識別システムの基本設計について述べる.3.3節ではシーケンス図を用いて商品識別システムの詳細設計を述べる.