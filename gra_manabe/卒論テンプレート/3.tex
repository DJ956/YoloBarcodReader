%第3章
%問題解決方法の選定:表を作成して問題解決方法を比較評価する.評価の理由と根拠をあげる(要データ,関連文献)
%解決方法の実行可能性の検証:自分がこの解決方法を選択した理由を述べる(3つの観点については本参照)


本章では,V字モデルによる商品識別システムの要求定義,基本設計,詳細設計について述べる.商品識別システムでは,全ての設計を通して,下記の3点を基本の評価軸とした.

\begin{itemize}
\item 従来の無人レジ店舗よりコストは抑えられるか.
\item 既存の中小店でも導入が容易か.
\item 従来のレジより簡単な動作で決済まで行えるか.
\end{itemize}


本章の構成について,3.1節ではユースケース図を用いて,商品識別システムの要求定義を述べる.3.2節ではクラス図を用いて,商品識別システムの基本設計について述べる.3.3節ではシーケンス図を用いて商品識別システムの詳細設計を述べる.
