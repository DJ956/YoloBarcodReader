%まえがき
%1-1要約:第一文-論文の趣旨,後1~2文①考えたこと②やったこと③結果の概要④成果の意義
%1-2問題設定(序論):研究をなぜ実施しなければならないのかを書く
%①あなたが解こうとする問題は何ですか?②その問題には必要性や需要はありますか?③あなたの取り組む問題は未解明で新規のものですか?④有用性はどれくらい見込めますか?
%1-3関連研究を引用.1)定石とすべき先行研究 2)定石化されていない部分の分岐状況
%各質問に対する回答パターンについては本参照



本論文では,Webカメラと超音波センサ,ロードセルなどのセンサを用い,安価なモビリティショッピング端末の開発を行う.

近年の日本において,少子高齢化の進行により,生産年齢人口は1995年をピークに減少に転じており,総人口も2008年をピークに減少に転じている\cite{population}.生産年齢人口の減少という問題は,スーパーマーケットにも顕著に表れており,人手不足対策のためセルフレジやセミセルフレジの導入,無人レジ店舗の導入が進んでいる.しかしながら,売り場規模別のセルフレジの設置率については,大規模店舗中心型が25\%を越えているのに対し,小規模や中規模の企業はそれぞれ7.1\%,7.4\%\cite{super}と低い状態となっている.また,今後のセルフレジの設置意向について,全国スーパーマーケット協会によるアンケートに,セルフレジを新たに設置したいと回答した割合が,都市圏では8.8\%なのに対し,地方圏では14.0\%\cite{super}と高くなっている.人手不足が続くなか,利用者のレジ待ち時間を解消するため,精算スピードが速くなるセルフ精算レジの導入意向が高くなっていることが分かる\cite{super}.人手不足の著しい地方圏のスーパーマーケットや小規模や中規模の企業へセルフレジの導入が進んでいない理由としては,コストがかかることが要因として挙げられる.無人レジ店舗においては数十台のカメラやセンサが必要であったり,商品すべてに独自のICタグを埋め込む必要があったりなど大きなコストを要するものとなっている.また,既存のスーパーマーケットにおいても,セルフレジの導入は費用の点で大きな負担がかかっているのが現実の問題としてあることが考えらえる.そこで,既存の無人レジ店舗のような複雑で高価なシステムではなく,小規模や中規模の企業でも導入できる安価なスマートモビリティレジシステムの作成を本研究の目的とした.この目標を達成するために,本研究では,Webカメラと超音波センサ,ロードセルなどのセンサを用い,安価なモビリティショッピング端末を開発することを目標とする.

具体的には,シングルボードコンピュータであるRaspberry PiとWebカメラ,各種センサを用い,商品の識別から決済に至るまでの一連の流れを行えるシステムの開発を行った.システムの開発では,V字モデルに従って,グループ(段原丞治,真鍋樹)でスマートモビリティレジシステムの開発を行った.要求分析,基本設計,詳細設計の際はUML(Unified Modeling Language)を用いた.

本論文の構成は下記のとおりである.第2章では本研究で用いる用語や研究方針,本システム全体の概要について述べる.第3章ではV字モデルに従った本システムの設計について述べる.第4章では,モビリティショッピング端末の実装と検証結果について述べる.第5章では実装・検証した本システムの評価を行い,考察を示す.第6章では本研究のまとめを行う.