%まえがき
%1-1要約:第一文-論文の趣旨,後1~2文①考えたこと②やったこと③結果の概要④成果の意義
%1-2問題設定(序論):研究をなぜ実施しなければならないのかを書く
%①あなたが解こうとする問題は何ですか?②その問題には必要性や需要はありますか?③あなたの取り組む問題は未解明で新規のものですか?④有用性はどれくらい見込めますか?
%1-3関連研究を引用.1)定石とすべき先行研究 2)定石化されていない部分の分岐状況
%各質問に対する回答パターンについては本参照



本論文では,Webカメラと超音波センサ,ロードセルなどのセンサを用い,安価な商品識別システムの開発を行った.

近年の日本において,少子高齢化の進行により,生産年齢人口は1995年をピークに減少に転じており,総人口も2008年をピークに減少に転じている\cite{population}.生産年齢人口の減少という問題は,スーパーマーケットにも顕著に表れており,働き手の数が少なくても経営できるように無人レジ店舗の導入や,セルフレジやセミセルフレジの導入が進んでいる.しかしながら,無人レジ店舗においては数十台のカメラやセンサが必要であったり,商品すべてに独自のICタグを埋め込む必要があったりなど大きなコストを要するものとなっている.また,既存のスーパーマーケットにおいても,セルフレジの導入は費用の点で大きな負担がかかっているのが現実である.そこで,既存の無人レジ店舗のような複雑で高価なシステムではなく,中小店でも導入できる安価なシステムの作成を本研究の目的とした.

本研究ではシングルボードコンピュータであるRaspberry PiとWebカメラ,各種センサを用い,商品の識別から決済に至るまでの一連の流れを行えるシステムの開発を行った.V字モデルに従って,グループ(段原丞治,真鍋樹)で商品識別システムの開発を行った.要求分析,基本設計,詳細設計の際はUMLを用いた.

本論文の構成は下記のとおりである.第2章では本研究で用いる用語や研究方針,商品識別システムの概要について述べる.第3章ではV字モデルに従った商品識別システムの設計について述べる.第4章では,商品識別システムの実装と検証結果について述べる.第5章では実装・検証したシステムの評価を行い,考察を示す.第6章では本研究のまとめを行う.