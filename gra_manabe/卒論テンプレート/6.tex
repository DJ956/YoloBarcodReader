%あとがき,結論
%5-1研究の成果(考えたこと,わかったこと)を書く(結論表),5-2成果の価値(学術的価値や実用的価値)を書く(先発性・明晰性・普遍性・可用性),5-3今後の課題(成果の限界とその克服のための戦略)を書く(アイデアの平凡さ・あいまいさ・狭さ・非実用性を検討

本論文ではセンシング技術を用いたモビリティショッピング端末の開発を行った.V字モデルに従い,要求分析,基本設計,詳細設計を行い,グループで役割分担をし開発を行った.システム全体の設計としては,UML図を用い方針を固めた.実装においては,優先度の高い機能を実装し,それぞれの単体テスト,結合テスト,総合テストを通して動作を確認し,評価を行った.検証を行う際,問題があった場合はそれぞれの設計に戻り,再度検証を行い,評価し,今後の課題について考察した.評価軸に沿った評価を行った後,今後の課題と解決策について考察をした.本システムは小規模店舗,中規模店舗を対象とし,コストを抑えたシステムとして先発性がある.本システムの開発が進めば,小規模店舗,中規模店舗での人手不足問題が解消される.本研究により本システムは今後拡張性があり,低コストで運用ができるシステムであることを確認した.