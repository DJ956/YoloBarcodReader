%第5章:評価・考察

本章では本システムを評価することにより,今後の課題について述べる.

まず,本システムを評価する.3.1節で設定した,3点の基本の評価軸より評価する.評価したものを書き表\ref{hyouka}に示す.

\begin{table}[htb]
\begin{center}
\caption{システムの評価}
\begin{tabular}{|c|c|} \hline
基本の評価軸 & 評価 \\ \hline \hline
従来のセルフレジよりコストは抑えられるか & 〇 \\
既存の中小店でも導入が容易か & △ \\
従来のセルフレジより簡単な動作で決済まで行えるか & △\\ \hline
\end{tabular}
\label{hyouka}
\end{center}
\end{table}

表\ref{hyouka}を上から順に説明する.従来のセルフレジよりコストは抑えられるかという評価軸について,2.2節で述べた表\ref{taisho}のスーパーマーケットを対象にして確認をする.Raspberry Piの価格は5,700円程度,各種センサと周辺機器の合計価格は3,500円程度のため,カゴにかかる合計価格は約9,200円とする.サーバと周辺機器にかかる価格を約150,000円とする.サーバ1台約150,000円とカゴ90個約828,000円とすると,本システムでかかる価格は約978,000円となり,従来のセルフレジとして2.2節で仮定した登録機1台と精算機7台の合計価格の約5\%程の価格となることが分かった.上記の理由から,従来のセルフレジよりコストを抑えられるとした.

次に,既存の中小店でも導入が容易かという評価軸においては,現段階では容易ではないため△とした.Raspberry Piや各種センサがしっかりと固定されておらず,誰でも導入ができるわけではないことが今後の課題となる.また,保守の点においてもセンサ類等がカゴに設置されるため,保守が難しくなるであろうという問題点もある.しかし,これからしっかり固定できるような状況ができれば,既存の買い物カゴに設置できる規模感であるため可能性がある.また,保守についても,抑えられたコストから少数の人員を割くことができ,解決ができるだろう.

次に,従来のセルフレジより簡単な動作で決済まで行えるかという評価軸においては,現時点では,商品のバーコードを読み取らせるために商品を回転させ,バーコードリーダを操作するような動作は必要はないが,バーコードをWebカメラに向けて台に置く必要があるため△とした.また,精度についても各センサの誤作動もあるため,完全であるとは言い切れない.簡単な動作で決済まで行えるかどうかについては問題点となる.しかしながら,バーコードがWebカメラに向けて置かれなかった場合についても,YOLOの開発が進めば商品のジャンルを判定できる可能性がある.また,ロードセルより重量のデータを得ることができるため,重量データと掛け合わせて商品を確定することができる可能性もある.画像識別の技術開発が進めば,バーコード情報だけでなく商品の情報を読み取ることができるという利点を持つWebカメラをモビリティショッピング端末に用いているため拡張性があるといえる.よって今後解決や開発が進めば可用かつ拡張性のあるシステムであると考えた.

