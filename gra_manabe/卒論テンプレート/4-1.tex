%第4-1章:実装
%手順(実験者のしたこと)


\section{実装}

第3章で述べた優先度の高いとした機能部分について実装を行った.本研究ではグループで開発を行った.サーバ側を段原丞治が,Raspberry Piと各種センサについてを筆者が実装した.Raspberry Piが制御した各種センサの実装環境について表\ref{jissou}に示す.


\begin{table}[htb]
\begin{center}
\caption{実装環境}
\begin{tabular}{|l|c|c|} \hline
実装環境 & センサ & 開発言語 \\ \hline \hline
Raspberry Pi 3 Model B & ロジクール ウェブカメラ C615 & python \\
Raspberry Pi 3 Model B & HY-SRF05超音波距離センサモジュール & python \\
Raspberry Pi 3 Model B & ロードセル シングルポイント(ビーム型)3kg & python \\
Raspberry Pi 3 Model B & SODIAL(R) 100 5mm & python \\ \hline
\end{tabular}
\label{jissou}
\end{center}
\end{table}


上記のセンサをショッピングバスケットに取り付け,実装を行った.ショッピングバスケットのサイズは33L,寸法はW510×D360×H240mmである.

これより,表\ref{jissou}に述べた各種センサの選定について述べる.


%もしかして以下の選定って3章に入れるべきかな

\subsection{Webカメラの選定}


まず,バーコードを読み取る装置の選定である.第3章に述べた基本の評価軸より,コストを抑えられるか,従来のセミセルフレジより簡単な動作で決済まで行えるか,かつバーコードを読み取ることができる装置,加えて拡張性のある装置かどうかを基準として選定した.コストを抑えているかつバーコードを読み取ることができる装置として,バーコードリーダー,Webカメラが挙げられる.バーコードリーダーとWebカメラのを比較した表を下記表\ref{came}に示す.


\begin{table}[htb]
\begin{center}
\caption{バーコードリーダーとWebカメラの比較}
\begin{tabular}{|l|c|c|c|c|} \hline
装置 & 価格 & 簡単さ &バーコードの読み取りの可否 & 拡張性 \\ \hline \hline
バーコードリーダー & 〇 & × & 〇 & × \\
Webカメラ & 〇 & △ & 〇 &  〇 \\ \hline
\end{tabular}
\label{came}
\end{center}
\end{table}


価格はどちらも2,000円台から購入できるため,2,000円のバーコードリーダやWebカメラをカゴ1個につき1台分購入したとしてもカゴ90個で180,000円とコストを抑えることが可能である.しかしながら簡単さにおいては,バーコードリーダーを使用する場合は従来のセミセルフレジの店員と同じ動きをしなければならないため,×の評価がつく.Webカメラにおいては,カゴ1個につき1台を導入したとすると定点カメラとなるため,カメラ側にバーコードを向けるという手間がかかる.拡張性についてはWebカメラの場合,バーコードをカメラに向けなくても商品の形状から商品もしくは商品の種類をを特定できるようになる等の例が挙げられた.上記より,本研究ではバーコードを読み取る装置としてWebカメラを選定した.

%カゴのどの位置・高さにどのように設置したか



次に,ユーザが商品をカゴに出し入れした際の動作の検知を行う装置の選定を行った.手もしくは商品がセンサの前を通ったかどうかを検知するセンサとして,焦電型赤外線センサー,距離センサー,超音波センサーが挙げられた.

%超音波センサの選定についてがんばれ
