%第4-1章:実装
%手順(実験者のしたこと)


\section{実装}

第3章で述べた優先度の高いとした機能部分について実装を行った.本研究ではグループで開発を行った.サーバ側を段原丞治が,Raspberry Piと各種センサについてを筆者が実装した.Raspberry Piが制御した各種センサの実装環境について表\ref{jissou}に示す.


\begin{table}[htb]
\begin{center}
\caption{実装環境}
\begin{tabular}{|l|c|c|} \hline
実装環境 & センサ & 開発言語 \\ \hline \hline
Raspberry Pi 3 Model B & ロジクール ウェブカメラ C615 & python \\
Raspberry Pi 3 Model B & HY-SRF05超音波距離センサモジュール & python \\
Raspberry Pi 3 Model B & ロードセル シングルポイント(ビーム型)3kg & python \\
Raspberry Pi 3 Model B & SODIAL(R) 100 5mm & python \\ \hline
\end{tabular}
\label{jissou}
\end{center}
\end{table}


上記のセンサをショッピングバスケットに取り付け,実装を行った.表\ref{jissou}に述べた各種センサの選定については3.2節に述べた.ショッピングバスケットのサイズは33L,寸法はW510×D360×H240mmである.



%選定を3-2へ移動した



%カゴのどの位置・高さにどのように設置したか